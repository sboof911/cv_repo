\documentclass[a4paper,11pt]{article}%
\usepackage[T1]{fontenc}%
\usepackage[utf8]{inputenc}%
\usepackage{lmodern}%
\usepackage{textcomp}%
\usepackage{lastpage}%
\usepackage[margin=1in]{geometry}%
\usepackage{enumitem}%
\usepackage{titlesec}%
\usepackage{needspace}%
\usepackage{graphicx}%
\usepackage[hidelinks]{hyperref}%
\usepackage{ragged2e}%
%
\titleformat{\section}{\large\bfseries}{}{0em}{}[\titlerule]%
\newcommand{\cvitem}[2]{\noindent\textbf{#1} \quad #2 \par}%
%
\begin{document}%
\normalsize%
\begin{center}%
{\Huge \textbf{MAACH Anass}}\\[0.3em]%
\'Etudiant en Architecture de l{\textquoteright}Information Num\'erique, sp\'ecialis\'e en Data/IA\\[0.3em]%
\href{mailto:anassmaachpro@gmail.com}{anassmaachpro@gmail.com}%
 \quad +212608603440 \quad %
\href{https://www.linkedin.com/in/anassmaach}{\raisebox{-0.2em}{\includegraphics[height=1.2em]{icons/linkedin.png}}/anassmaach} \quad %
\href{https://github.com/sboof911}{\raisebox{-0.2em}{\includegraphics[height=1.2em]{icons/github.png}}/sboof911}%
\end{center}%
\section*{Profile}%
Passionn\'e par l{\textquoteright}intelligence artificielle, avec un fort int\'er\^et pour l{\textquoteright}apprentissage par renforcement et l{\textquoteright}automatisation. Comp\'etent en Python et dans ses frameworks pour la data science et l{\textquoteright}IA, notamment Pandas, PyArrow, TensorFlow, PyTorch et Keras.\newline Exp\'eriment\'e avec diverses architectures d{\textquoteright}IA telles que les LLM, CNN, RNN, MLP et les mod\`eles de RL.\newline \newline Aussi comp\'etent en web scraping (Requests, BeautifulSoup, Selenium), et ayant de l{\textquoteright}exp\'erience en C, C++, TypeScript et Lua.\newline D\'esireux de continuer \`a progresser dans le domaine de l{\textquoteright}IA et de contribuer \`a des projets innovants.%
\section*{Education}%
\noindent\makebox[3.5cm][l]{\textbf{2023--2025}}\begin{minipage}[t]{\dimexpr\linewidth-3.5cm\relax}Architecte de l{\textquoteright}Information Num\'erique in IA et Data, 1337-UM6P\end{minipage}\par%
\noindent\makebox[3.5cm][l]{\textbf{2019--2023}}\begin{minipage}[t]{\dimexpr\linewidth-3.5cm\relax}D\'eveloppeur Junior in G\'enie Logiciel, 1337-UM6P\end{minipage}\par%
\noindent\makebox[3.5cm][l]{\textbf{2016--2019}}\begin{minipage}[t]{\dimexpr\linewidth-3.5cm\relax}Dipl\^ome Universitaire de 2 ans in Math\'ematiques et Informatique, Universit\'e IBN ZOHR\end{minipage}\par%
\noindent\makebox[3.5cm][l]{\textbf{2016--2016}}\begin{minipage}[t]{\dimexpr\linewidth-3.5cm\relax}Baccalaur\'eat in Sciences Math\'ematiques, Lyc\'ee Ibno Maja\end{minipage}\par%
\section*{Professional Experience}%
\textbf{Stagiaire D\'eveloppeur Python, Vanguard} \hfill Mai 2023 -- Oct 2023%
\begin{itemize}[leftmargin=*]%
\item Contribu\'e au d\'eveloppement de Pyccel, un compilateur Python vers C.%
\item Impl\'ement\'e le support des classes dans le projet.%
\item Particip\'e \`a l{\textquoteright}optimisation et \`a d{\textquoteright}autres t\^aches de d\'eveloppement.%
\end{itemize}%
\needspace{3\baselineskip}%
\section*{Deep Learning}%
\needspace{3\baselineskip}%
\noindent \textbf{Fried Eggs} - \href{https://github.com/sboof911/Fried-eggs}{{\raisebox{-0.2em}{\includegraphics[height=1.2em]{icons/github.png}}}}%
\begin{itemize}[leftmargin=2em,parsep=0pt,topsep=1em]%
\item[] \textbf{Tools:} Python, Deep Learning, CNN, Instance Segmentation, Object Detection.%
\item Annot\'e des images avec CVAT et pr\'epar\'e des jeux de donn\'ees pour des t\^aches de vision par ordinateur.%
\item Pr\'etrait\'e et entra{\^\i}n\'e des r\'eseaux de neurones convolutifs (CNN) pour la d\'etection et la segmentation d'objets.%
\item Construit un pipeline complet pour pr\'edire et classifier de nouvelles images au-del\`a du jeu d'entra{\^\i}nement.%
\end{itemize}%
\needspace{3\baselineskip}%
\noindent \textbf{Leaffliction} - \href{https://github.com/sboof911/Leaffliction}{{\raisebox{-0.2em}{\includegraphics[height=1.2em]{icons/github.png}}}}%
\begin{itemize}[leftmargin=2em,parsep=0pt,topsep=1em]%
\item[] \textbf{Tools:} Python, PyTorch, ResNet-18, Transfer Learning, Image Augmentation.%
\item D\'evelopp\'e un mod\`ele de d\'etection de maladies des plantes en utilisant le transfert d'apprentissage avec ResNet-18.%
\item Appliqu\'e l'augmentation de donn\'ees (redimensionnement, niveaux de gris, flou) pour am\'eliorer la robustesse du mod\`ele.%
\item S\'epar\'e l'entra{\^\i}nement et la pr\'ediction dans des scripts modulaires pour une r\'eutilisation facile.%
\end{itemize}%
\needspace{2\baselineskip}%
\noindent \textbf{Game Bot (Personal Project)}%
\begin{itemize}[leftmargin=2em,parsep=0pt,topsep=1em]%
\item[] \textbf{Tools:} Python, Sockets, Reinforcement Learning.%
\item Cr\'e\'e une interface qui connecte un bot Python personnalis\'e \`a un jeu en ligne via des sockets pour une communication en temps r\'eel.%
\item Automatisation du jeu en int\'egrant des mod\`eles d'apprentissage par renforcement, permettant au bot de prendre des d\'ecisions et de jouer de fa\c{c}on autonome.%
\end{itemize}%
\needspace{3\baselineskip}%
\section*{Machine Learning}%
\needspace{3\baselineskip}%
\noindent \textbf{MySpotify} - \href{https://github.com/sboof911/MySpotify}{{\raisebox{-0.2em}{\includegraphics[height=1.2em]{icons/github.png}}}}%
\begin{itemize}[leftmargin=2em,parsep=0pt,topsep=1em]%
\item[] \textbf{Tools:} Python, Pandas, Word2Vec, MLPClassifier, Alternating Least Squares (ALS).%
\item Construit un moteur de recommandation int\'egrant plusieurs strat\'egies : baseline, filtrage par contenu et filtrage collaboratif.%
\item Impl\'ement\'e des embeddings Word2Vec et des classificateurs MLP pour mod\'eliser la similarit\'e musicale et les pr\'ef\'erences des utilisateurs.%
\item D\'evelopp\'e un filtrage collaboratif bas\'e sur ALS pour des recommandations personnalis\'ees.%
\end{itemize}%
\needspace{3\baselineskip}%
\noindent \textbf{Multilayer Perceptron} - \href{https://github.com/sboof911/Multilayer-Perceptron}{{\raisebox{-0.2em}{\includegraphics[height=1.2em]{icons/github.png}}}}%
\begin{itemize}[leftmargin=2em,parsep=0pt,topsep=1em]%
\item[] \textbf{Tools:} Python, NumPy, Matplotlib, Neural Networks.%
\item Impl\'ement\'e un perceptron multicouche \`a propagation avant depuis z\'ero pour classifier des donn\'ees sur le cancer du sein.%
\item Cr\'e\'e des modules d'entra{\^\i}nement, d'\'evaluation et de visualisation incluant les courbes de pr\'ecision et de perte.%
\item Con\c{c}u un pipeline de pr\'ediction modulaire pour valider des cas de test r\'eels.%
\end{itemize}%
\needspace{3\baselineskip}%
\noindent \textbf{Total-perspective-vortex} - \href{https://github.com/sboof911/total-perspectivevortex}{{\raisebox{-0.2em}{\includegraphics[height=1.2em]{icons/github.png}}}}%
\begin{itemize}[leftmargin=2em,parsep=0pt,topsep=1em]%
\item[] \textbf{Tools:} Python, Data Processing, EEG, Logistic Regression, MLP, Random Forest.%
\item Pr\'etrait\'e et visualis\'e des donn\'ees EEG pour les pr\'eparer \`a des applications d'apprentissage automatique.%
\item Construit et compar\'e plusieurs mod\`eles, dont la r\'egression logistique, MLP et Random Forest, pour \'evaluer la pr\'ecision des pr\'edictions.%
\item D\'evelopp\'e un pipeline pour pr\'edire les actions des utilisateurs \`a partir des signaux EEG, explorant des applications en interaction cerveau-machine.%
\end{itemize}%
\needspace{3\baselineskip}%
\section*{NLP / Text Analysis}%
\needspace{3\baselineskip}%
\noindent \textbf{Understanding Customer} - \href{https://github.com/sboof911/Understanding-customer}{{\raisebox{-0.2em}{\includegraphics[height=1.2em]{icons/github.png}}}}%
\begin{itemize}[leftmargin=2em,parsep=0pt,topsep=1em]%
\item[] \textbf{Tools:} Python, Jupyter Notebook, RNN, LSTM, Transformer, BERT, Falconsai intent\_classification.%
\item Con\c{c}u des exp\'eriences d'apprentissage profond pour la classification d'intentions avec des mod\`eles RNN, LSTM et Transformer.%
\item Ajust\'e le mod\`ele Falconsai/intent\_classification sur des jeux de donn\'ees personnalis\'es pour am\'eliorer la pr\'ecision.%
\item Compar\'e les pipelines de pr\'etraitement et les architectures pour \'evaluer la performance des mod\`eles.%
\end{itemize}%
\needspace{3\baselineskip}%
\noindent \textbf{Tweets} - \href{https://github.com/sboof911/tweets}{{\raisebox{-0.2em}{\includegraphics[height=1.2em]{icons/github.png}}}}%
\begin{itemize}[leftmargin=2em,parsep=0pt,topsep=1em]%
\item[] \textbf{Tools:} Python, Natural Language Processing, TF-IDF, Word2Vec.%
\item Explor\'e des techniques NLP incluant bag-of-words, TF-IDF, stemming, lemmatisation et suppression des stopwords.%
\item Impl\'ement\'e des mod\`eles de similarit\'e avec la distance cosinus et l'analyse n-gram.%
\item Int\'egr\'e des embeddings word2vec pour am\'eliorer la compr\'ehension s\'emantique.%
\end{itemize}%
\needspace{3\baselineskip}%
\section*{Time Series \& Data Analysis}%
\needspace{3\baselineskip}%
\noindent \textbf{City Life} - \href{https://github.com/sboof911/City-Life}{{\raisebox{-0.2em}{\includegraphics[height=1.2em]{icons/github.png}}}}%
\begin{itemize}[leftmargin=2em,parsep=0pt,topsep=1em]%
\item[] \textbf{Tools:} Python, GeoPandas, Clustering, Geospatial Analytics.%
\item R\'ealis\'e une analyse g\'eospatiale sur des jeux de donn\'ees de mobilit\'e urbaine.%
\item Appliqu\'e des techniques de clustering pour identifier des motifs dans la distribution des trajets en ville.%
\item Construit des cartes interactives pour visualiser les r\'esultats avec GeoPandas.%
\end{itemize}%
\needspace{3\baselineskip}%
\noindent \textbf{Uber} - \href{https://github.com/sboof911/Uber}{{\raisebox{-0.2em}{\includegraphics[height=1.2em]{icons/github.png}}}}%
\begin{itemize}[leftmargin=2em,parsep=0pt,topsep=1em]%
\item[] \textbf{Tools:} Python, Pandas, Time Series Analysis, ARIMA, Exponential Smoothing.%
\item R\'ealis\'e une analyse de s\'eries temporelles sur les donn\'ees de trajets Uber pour pr\'evoir la demande.%
\item Impl\'ement\'e des mod\`eles TES (Triple Exponential Smoothing) et ARIMA.%
\item Cr\'e\'e des variables telles que le nombre de trajets par heure, jour et mois.%
\end{itemize}%
\needspace{3\baselineskip}%
\section*{Games \& Algorithm}%
\needspace{3\baselineskip}%
\noindent \textbf{Gomoku} - \href{https://github.com/sboof911/Gomoku}{{\raisebox{-0.2em}{\includegraphics[height=1.2em]{icons/github.png}}}}%
\begin{itemize}[leftmargin=2em,parsep=0pt,topsep=1em]%
\item[] \textbf{Tools:} Python, TypeScript, React, Minimax Algorithm, Alpha-Beta Pruning, Memoization.%
\item D\'evelopp\'e un jeu complet de Gomoku avec backend (Python) et frontend (TypeScript/React).%
\item Impl\'ement\'e un adversaire IA utilisant l'algorithme Minimax avec \'elagage alpha-b\^eta et heuristiques.%
\item Optimis\'e les performances en introduisant la m\'emorisation pour \'eviter les calculs redondants.%
\end{itemize}%
\needspace{3\baselineskip}%
\section*{Software Projects}%
\needspace{3\baselineskip}%
\noindent \textbf{Ft\_transcendence} - \href{https://github.com/sboof911/ft_transcendence}{{\raisebox{-0.2em}{\includegraphics[height=1.2em]{icons/github.png}}}}%
\begin{itemize}[leftmargin=2em,parsep=0pt,topsep=1em]%
\item[] \textbf{Tools:} TypeScript, NestJS, React, PostgreSQL, WebSockets, Docker.%
\item D\'evelopp\'e une application web full-stack avec authentification, chat et jeu Pong multijoueur.%
\item Backend construit avec NestJS et PostgreSQL, frontend avec React.%
\item Int\'egr\'e WebSockets pour les interactions en temps r\'eel et containeris\'e avec Docker.%
\end{itemize}%
\needspace{3\baselineskip}%
\noindent \textbf{Webserv} - \href{https://github.com/sboof911/webserv}{{\raisebox{-0.2em}{\includegraphics[height=1.2em]{icons/github.png}}}}%
\begin{itemize}[leftmargin=2em,parsep=0pt,topsep=1em]%
\item[] \textbf{Tools:} C++, HTTP/1.1, Sockets.%
\item D\'evelopp\'e un serveur HTTP enti\`erement fonctionnel en C++ supportant HTTP/1.1.%
\item G\'er\'e la programmation des sockets, le parsing des requ\^etes et la g\'en\'eration des r\'eponses.%
\item Support\'e l'ex\'ecution CGI et test\'e la compatibilit\'e avec de vrais navigateurs.%
\end{itemize}%
\needspace{2\baselineskip}%
\noindent \textbf{Minishell} - \href{https://github.com/sboof911/minishell}{{\raisebox{-0.2em}{\includegraphics[height=1.2em]{icons/github.png}}}}%
\begin{itemize}[leftmargin=2em,parsep=0pt,topsep=1em]%
\item[] \textbf{Tools:} C, Unix System Calls, Parsing \& Tokenization, Signal Handling.%
\item Recr\'e\'e un shell simplifi\'e de type Bash en C depuis z\'ero.%
\item Impl\'ement\'e les fonctionnalit\'es principales : commandes int\'egr\'ees (echo, cd, pwd, export, unset, env, exit), expansion des variables d'environnement.%
\end{itemize}%
\section*{Skills}%
\begin{itemize}[leftmargin=*]%
\item \textbf{Langages de Programmation}: Python, C++, C, Lua%
\item \textbf{Frameworks}: TensorFlow, PyTorch, FastAPI%
\item \textbf{IA/ML}: CNN, LLM, RL, MLP, Agents%
\item \textbf{Outils}: Git, Docker, Linux, Kubernetes%
\end{itemize}%
\section*{Languages}%
Anglais (Courant)%
\newline%
%
Fran\c{c}ais (Courant)%
\newline%
%
Arabe (Langue Maternelle)%
\newline%
%
\section*{Hobbies}%
Composer/Jouer de la Musique%
\newline%
%
Cyclisme%
\newline%
%
Billard%
\newline%
%
\end{document}